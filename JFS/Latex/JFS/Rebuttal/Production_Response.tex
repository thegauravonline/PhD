\documentclass[onecolumn,a4paper,amsmath,amssym,pre]{revtex4}
\usepackage{graphicx}% Include figure files
\usepackage{dcolumn}% Align table columns on decimal point
\usepackage{subfigure}
\usepackage{bm}% bold math
\bibliographystyle{pf}
\usepackage{setspace}
\usepackage{fancyhdr}
\usepackage{amsmath}
\usepackage{color}
\usepackage{epstopdf}
\usepackage{booktabs}
\usepackage{xcolor}
\usepackage{hyperref}
\usepackage{ulem}

\addtolength{\parskip}{0pt}
\addtolength{\floatsep}{0pt}
\addtolength{\textfloatsep}{0pt}
\addtolength{\abovecaptionskip}{0pt}
\addtolength{\belowcaptionskip}{0pt}
%\newcommand{\ra}[1]{\renewcommand{\arraystretch}{#1}
\newcommand{\volume}{{\ooalign{\hfil$V$\hfil\cr\kern0.08em--\hfil\cr}}}




\begin{document}
    

\begin{tabular}{ r | l }
  \hline            
   \textbf{Manuscript details}:& EX11802 \\	
   \textbf{Title}: & Wall mounted flexible plates in a two-dimensional channel trigger early flow instabilities\\  
  \textbf{Authors}:  & Gaurav Singh and Rajaram Lakkaraju \\
  \hline  
\end{tabular}

    
    Dear Sir/Madam, \\
    
    Greetings. 
    Our sincere thanks to you for accepting our manuscript (EX11802) entitled "Wall-mounted flexible plates in a two-dimensional channel trigger early flow instabilities". As per your direction about improving figures compatibility for the print version, we have replaced the following figures with distinguishable shades in monochrome. The corresponding captions and descriptions have also been updated as per the standards of the journal. We again appreciate the kindness of the editor in helping to improve the manuscript. Thank you again for the consideration.
    
    
    
    \section{Revised Figures and captions}
    
    \begin{enumerate}    
    	
    	\item \textcolor{red}{Figure 3}\\
    	\textbf{Revised Caption}: Single plate dynamics in a channel flow: (a) Filament position on x-y plane at different times for $Re=200$. Thick line for the steady-state shape. (b) tip deflection of a single plate ($\delta_S$) with time. Light gray line for $Re=200$ and blue (dark gray) line for $Re=3200$. (c) steady-state tip deflection ($\overline{\delta}_S$) vs. $Re$. A least square fit gives $(\overline{\delta}_S/l)\propto Re^{0.018}$.
    	
    	
    	\item \textcolor{red}{Figure 4}\\
    	\textbf{Revised Caption}: Normalized tip deflection ($\delta/\overline{\delta}$) vs. dimensionless time ($t/T$) in case of a single plate anchored in a channel flow: (a) for $Re=200$, and (b) for $Re=1600$. Here, $\overline{\delta}$ is the steady-state tip deflection. Dotted line represents the SMD theory and dark solid line represents the FSI simulations. Insets for both initial transients and steady-state response.
    	
    	
    	\item \textcolor{red}{Figure 5}\\
    	\textbf{Revised Caption}: Two plate dynamics in a channel flow for the plate gap $d/h=2$: (a) the plate shapes on x-y plane at different times for $Re=200$. Thick line for the steady-state shape. (b) Tip deflection of a plate 1 with time, and (c) tip deflection of a plate 2 with time.  Light gray line for $Re=200$ and blue (dark gray) line for $Re=3200$. Inset: tip deflection at plate 2 shows a beat-like pattern.
    	
    	\item \textcolor{red}{Figure 7}\\
    	\textbf{Revised Caption}: Normalized tip deflections vs. dimensionless time: Top panels (a, b and c) for $Re=200$ and bottom panels (d, e and f) for $Re=1600$. The plate gaps are $d/h=2$ (in panels a and d), $d/h=1$ (in panels b and e), and $d/h=0.5$ (in panels c and f). Dotted line for the SMD theory, black solid line for the plate 1 and green (gray) line for the plate 2. Insets for tip deflections both at the initial transients and at the steady-state.
    		
    	\item \textcolor{red}{Figure 11}\\
    	\textbf{Revised Caption}: Time-averaged center-line Reynolds number ($Re_c$) along the channel's length. Panels: (a) when the inlet flow $Re=800$ and (b) when inlet flow $Re=3200$. In panels c and d, $Re_c$ in a double plate configuration is normalized with $Re_c$ of a single plate configuration ($Re_{cS}$). To guide eye, dotted lines show the flow blockage zone.
    	
    	\item \textcolor{red}{Figure 12}\\
    	\textbf{Revised Caption}: Cross-wise velocity vs. time measured at point {\fontfamily{phv}\normalsize{X}}$(x=7h,y=0)$ in the downstream for: (a) a single plate case, and (b) a double plate case with $d/h=2$. Spectral energy $E$ vs. Strouhal number ($St$) shown in panels (c) for a single plate case and (d) for a double plate case with $d/h=2$. Insets for zoom-up on crosswise velocity time series. Note, the velocity signal for single plate case is $\approx0$ for both $Re=200$ and $Re=290$.  	
    	    	
    	\item \textcolor{red}{Figure 16}\\
    	\textbf{Revised Caption}: Dominant POD modes in cases of a single plate and double plate configurations: Left panels for $Re=800$, and right panels for $Re=3200$. The dominant spatial mode is represented by velocity vectors and it is color mapped by streamwise velocity fluctuation. Dark blue for $-0.02 {v_{\infty}}$ and light red for $0.02{v_{\infty}}$.
    	
    	\item \textcolor{red}{Figure 18}\\
    	\textbf{Revised Caption}:(a) Probability density function ($\mathcal{F}$) of the kinetic energy dissipation, and (b) $\mathcal{F}$ vs. $\epsilon/\overline{\epsilon}_{rms}$. Normal Gaussian curve in thin dashes and a stretched exponential in thick dashes are shown. Inset for $\overline{\epsilon}_{rms}$ as a function of plate gap $d/h$. Simulations are for $Re=3200$ and data statistics taken at the channel core in Region A.
    
    	\textit{In reference to the updates in figure 18, we have updated the notation for Probability density function from $PDF$ to $\mathcal{F}$ at all instances in the text, including in the "abstract".}
    	
    	
    	
    \end{enumerate}
    
    
    
    \vspace{1cm} 
    \underline{Yours sincerely}, \\
    \vspace{0.3cm} \\
    Rajaram Lakkaraju  \\
    \today \\
    IIT Kharagpur, India.
    
\end{document}